\documentclass[a4paper, 12pt]{article}
\usepackage{comment} 
\usepackage{lipsum} 
\usepackage{fullpage} 
\usepackage[a4paper, total={7in, 10in}]{geometry}
\usepackage{setspace}
\onehalfspacing % полуторный интервал для всего текста

\usepackage[fleqn]{amsmath}
\usepackage{mathtools,amsmath}
\usepackage{amssymb,amsthm} % assumes amsmath package installed
\newtheorem{theorem}{Theorem}
\newtheorem{corollary}{Corollary}
\usepackage{graphicx}
\usepackage{tikz}
\usetikzlibrary{arrows}
\usepackage{verbatim}
\usepackage{float}
\usepackage{tikz}
\usetikzlibrary{automata,positioning}
\usepackage{pgfplots}
\usetikzlibrary{shapes,arrows}
\usetikzlibrary{arrows,calc,positioning}
\tikzset{
	block/.style = {draw, rectangle,
		minimum height=1cm,
		minimum width=1.5cm},
	input/.style = {coordinate,node distance=1cm},
	output/.style = {coordinate,node distance=4cm},
	arrow/.style={draw, -latex,node distance=2cm},
	pinstyle/.style = {pin edge={latex-, black,node distance=2cm}},
	sum/.style = {draw, circle, node distance=1cm},
}
\usepackage{xcolor}
\usepackage{mdframed}
\usepackage[shortlabels]{enumitem}
\usepackage{indentfirst}
\usepackage{hyperref}
\usepackage{wrapfig}
\usepackage{tcolorbox}

%% Работа с языком
\usepackage[utf8]{inputenc}
\usepackage[english]{babel}
\usepackage{booktabs}
\usepackage{pgfplots}
\pgfplotsset{width=9cm, height=5cm, compat=1.9}

\renewcommand{\thesubsection}{\thesection.\alph{subsection}}

\newenvironment{problem}[2][Задача]
{ \begin{mdframed}[backgroundcolor=gray!20] \textbf{#1 #2} \\}
	{ \end{mdframed}}

% Define solution environment
\newenvironment{solution}
{\textit{Решение:}}
{}
\renewcommand{\qed}{\quad\qedsymbol}


\newcommand{\R}{\mathbb{R}}
\newcommand{\E}{\mathbb{E}}
\renewcommand{\P}{\mathbb{P}}
\DeclareMathOperator{\Var}{\mathbb{V}ar}
\DeclareMathOperator{\AsVar}{As.\mathbb{V}ar}
\DeclareMathOperator{\Cov}{\mathbb{C}ov}
\DeclareMathOperator{\Corr}{\mathbb{C}orr}
\DeclareMathOperator{\sign}{sign}
\DeclareMathOperator{\tr}{tr}
\DeclareMathOperator{\rank}{rank}
\DeclareMathOperator{\argmax}{argmax}
\DeclareMathOperator{\MSE}{MSE}


\author{Nikolai Averianov, Sasha Plakhin}
\title{Reading list}

\begin{document}

\maketitle

Main textbook for reading is 
Arbitrage Theory in Continuous Time by Tomas Björk
\section*{Simulations}
\begin{enumerate}
	\item \href{http://www.columbia.edu/~mh2078/MonteCarlo/MCS_SDEs.pdf}{Simulating Stochastic Differential Equations}
	
	\item \href{https://arxiv.org/pdf/1004.0646.pdf}{An introduction to SDE simulation}
	
	\item \href{https://epubs.siam.org/doi/pdf/10.1137/S0036144500378302}{An Algorithmic Introduction to Numerical Simulation of Stochastic Differential Equations}
	
	\item \href{http://www.cs.fsu.edu/~mascagni/Petersen_SDE.pdf}{Introduction to the Numerical Simulation of Stochastic Differential Equations with Examples}
	
	\item \href{https://www.researchgate.net/publication/4053873_Efficient_simulation_of_Gamma_and_variance-Gamma_processes}{Efficient simulation of Gamma and variance-Gamma processes}
\end{enumerate}

\section*{Levy processes}
\begin{enumerate}
	\item \href{https://www.stats.ox.ac.uk/~winkel/ms3b10.pdf}{Levy Processes and Finance course}
	
	\item \href{https://arxiv.org/pdf/0804.0482.pdf}{An introfuction to Levy processes with applications in finance}
\end{enumerate}

\section*{Statistics}
\begin{enumerate}
	\item \href{http://www.thierry-roncalli.com/download/copula-survey.pdf}{Copulas for Finance A Reading Guide and Some Applications}
\end{enumerate}

\section*{ML/DL applications}
\begin{enumerate}
\item \href{http://thierry-roncalli.com/download/rbm_gan_backtesting.pdf}{Improving the Robustness of Trading Strategy Backtesting with Boltzmann Machines and Generative Adversarial Networks}

\item \href{https://arxiv.org/pdf/2106.00123.pdf}{Deep Reinforcement Learning in Quantitative Algorithmic Trading: A Review}
\end{enumerate}

\section*{Misc}
\begin{enumerate}
	
	\item \href{https://www.maths.usyd.edu.au/u/UG/SM/MATH3075/r/Joshi_2008.pdf}{On becoming a quant}
	\item \href{https://engineering.nyu.edu/sites/default/files/2021-10/How_I_Became_a_Quant\%20\%281\%29.pdf}{How I Became a Quant}
\end{enumerate}




\end{document}



