\documentclass[a4paper, 12pt]{article}
\usepackage{comment} 
\usepackage{lipsum} 
\usepackage{fullpage} 
\usepackage[a4paper, total={7in, 10in}]{geometry}
\usepackage{setspace}
\onehalfspacing % полуторный интервал для всего текста

\usepackage[fleqn]{amsmath}
\usepackage{mathtools,amsmath}
\usepackage{amssymb,amsthm} % assumes amsmath package installed
\newtheorem{theorem}{Theorem}
\newtheorem{corollary}{Corollary}
\usepackage{graphicx}
\usepackage{tikz}
\usetikzlibrary{arrows}
\usepackage{verbatim}
\usepackage{float}
\usepackage{tikz}
\usetikzlibrary{automata,positioning}
\usepackage{pgfplots}
\usetikzlibrary{shapes,arrows}
\usetikzlibrary{arrows,calc,positioning}
\tikzset{
	block/.style = {draw, rectangle,
		minimum height=1cm,
		minimum width=1.5cm},
	input/.style = {coordinate,node distance=1cm},
	output/.style = {coordinate,node distance=4cm},
	arrow/.style={draw, -latex,node distance=2cm},
	pinstyle/.style = {pin edge={latex-, black,node distance=2cm}},
	sum/.style = {draw, circle, node distance=1cm},
}
\usepackage{xcolor}
\usepackage{mdframed}
\usepackage[shortlabels]{enumitem}
\usepackage{indentfirst}
\usepackage{hyperref}
\usepackage{wrapfig}
\usepackage{tcolorbox}

%% Работа с языком
\usepackage[utf8]{inputenc}
\usepackage[english]{babel}
\usepackage{booktabs}
\usepackage{pgfplots}
\pgfplotsset{width=9cm, height=5cm, compat=1.9}

\renewcommand{\thesubsection}{\thesection.\alph{subsection}}

\newtheoremstyle{problemstyle}  % <name>
{3pt}                                               % <space above>
{3pt}                                               % <space below>
{\normalfont}                               % <body font>
{}                                                  % <indent amount}
{\bfseries\itshape}                 % <theorem head font>
{\normalfont\bfseries:}         % <punctuation after theorem head>
{.5em}                                          % <space after theorem head>
{}                                                  % <theorem head spec (can be left empty, meaning `normal')>
\theoremstyle{problemstyle}

\newtheorem{problem}{Problem}[section]

% Define solution environment
\newenvironment{solution}
{\textit{Solution:}}
{}
\renewcommand{\qed}{\quad\qedsymbol}


\newcommand{\R}{\mathbb{R}}
\newcommand{\E}{\mathbb{E}}
\renewcommand{\P}{\mathbb{P}}
\DeclareMathOperator{\Var}{\mathbb{V}ar}
\DeclareMathOperator{\AsVar}{As.\mathbb{V}ar}
\DeclareMathOperator{\Cov}{\mathbb{C}ov}
\DeclareMathOperator{\Corr}{\mathbb{C}orr}
\DeclareMathOperator{\sign}{sign}
\DeclareMathOperator{\tr}{tr}
\DeclareMathOperator{\rank}{rank}
\DeclareMathOperator{\argmax}{argmax}
\DeclareMathOperator{\MSE}{MSE}





% магия для автоматических гиперссылок задача-решение
\newlist{myenum}{enumerate}{3}
% \newcounter{problem}[chapter] % нумерация задач внутри глав


% \setcounter{tocdepth}{1} % в оглавление оставляем уровень 1




\author{Nick Averyanov, Sasha Plakhin}
\title{Financial math problems solutions}

\begin{document}

\maketitle

\newpage
\tableofcontents{}

\newpage
\section{Binomial model}

\section{Itô's lemma}
\begin{problem}
$h(\cdot)$ -- is a harmonic function if:
$$
\sum_{i=1}^n \cfrac{\partial^2 h}{\partial x_i^2} = 0.
$$
$h(\cdot) $ -- is a subharmonic function if:
$$
\sum_{i=1}^n \cfrac{\partial^2 h}{\partial x_i^2} \geq 0.
$$
Prove that for independent  Wiener processes $W_1, \ldots, W_n$ and a processes $X$ is defined by the formula: $X(t) = h(W_1(t), \ldots, W_n(t))$. Show that if $h$ is harmonic (subharmonic) $\Rightarrow$  $X$ is a martingale (submartingale).
\end{problem}

\begin{solution}
	Applying multidimensional version of Ito's lemma to the function $h$ we can obtain:
	$$
	dX = \sum_{i=1}^n \cfrac{\partial h}{\partial x_i} dW_i + \cfrac{1}{2} \sum_{i=1}^n \cfrac{\partial^2 h}{\partial x_i^2} dt
	$$
	Equivalently, we can rewrite the equation as:
	$$
	h(W_1(t), \ldots, W_n(t)) ) = h(\boldsymbol{0}) + \int_0^t  \sum_{i=1}^n \cfrac{\partial h}{\partial x_i} dW_i + \int_0^t\cfrac{1}{2} \sum_{i=1}^n \cfrac{\partial^2 h}{\partial x_i^2} dt
	$$
	\begin{multline*}
		\E[X(t)| \mathcal{F}_s] = h(\boldsymbol{0})  + \E\bigg[ \int_0^s  \sum_{i=1}^n \cfrac{\partial h}{\partial x_i} dW_i + \int_0^s\cfrac{1}{2} \sum_{i=1}^n \cfrac{\partial^2 h}{\partial x_i^2} dt + \int_s^t  \sum_{i=1}^n \cfrac{\partial h}{\partial x_i} dW_i + \int_s^t\cfrac{1}{2} \sum_{i=1}^n \cfrac{\partial^2 h}{\partial x_i^2} dt | \mathcal{F}_s \bigg ] =  \\
		X(s) + \E \bigg[ \int_s^t  \sum_{i=1}^n \cfrac{\partial h}{\partial x_i} dW_i + \int_s^t\cfrac{1}{2} \sum_{i=1}^n \cfrac{\partial^2 h}{\partial x_i^2} dt | \mathcal{F}_s \bigg ] = X(s) + \E \bigg[ \int_s^t\cfrac{1}{2} \sum_{i=1}^n \cfrac{\partial^2 h}{\partial x_i^2} dt | \mathcal{F}_s \bigg ]
	\end{multline*}
	Looking at the last term we see that there is a sum of second order partial derivatives. If $h$ is harmonic we have $\E[X(t)|\mathcal{F}_s] = X(s)$ a.s. If  $h$ is subharmonic we have $\E[X(t)|\mathcal{F}_s] \geq X(s)$ a.s. And thus we have proven the statement.

\end{solution}

\begin{problem}
	Show that $dW_1 dW_2 = 0 $ for two independent Brownian motions.
\end{problem}
Let $\Delta W_i(t_k)  = W_i(t_k) - W_i(t_{k-1})$.
Define $Q_n$:
$$
Q_n = \sum_{k=1}^n  \Delta W_1(t_k) \Delta W_2(t_k),
$$
where $0 = t_0 < t_1 < \ldots <t_n = t$. Now we just have to show that $Q_n$ converges to $0$ in $L^2$ as the norm of the partiotion goes to $0$. We have:
$$
\E[Q_n] = \E \sum_{k=1}^n  \Delta W_1(t_k) \Delta W_2(t_k)  =  \sum_{k=1}^n  \E[ \Delta W_1(t_k) \Delta W_2(t_k)] = 0
$$
\begin{multline*}
\E[Q_n^2] = \Var[Q_n]  = \Var\bigg[ \sum_{k=1}^n  \Delta W_1(t_k) \Delta W_2(t_k)  \bigg] = 
 \sum_{k=1}^n \Var[\Delta W_1(t_k) \Delta W_2(t_k)] = \\
 \sum_{k=1}^n  \E[\Delta W_1(t_k)^2 \Delta W_2(t_k)^2] = \sum_{k=1}^n  \E[\Delta W_1(t_k)^2] \E[\Delta W_2(t_k)^2]  = \\ \sum_{k=1}^n  (\Delta t_k)^2 \leq 
 \max_k \{\Delta t_k \} \sum_{k=1}^n \Delta t_k =  \max_{k \in \{, \ldots n\}} \{\Delta t_k \} t \to 0, n \to \infty
\end{multline*}

\section{Martingales}

\section{Partial differential equations}

\section{Stochastic differential equations}

\section{Black-Scholes}




\end{document}



