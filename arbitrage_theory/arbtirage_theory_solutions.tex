\documentclass[a4paper, 12pt]{article}
\usepackage{comment} 
\usepackage{lipsum} 
\usepackage{fullpage} 
\usepackage[a4paper, total={7in, 10in}]{geometry}
\usepackage{setspace}
\onehalfspacing % полуторный интервал для всего текста

\usepackage[fleqn]{amsmath}
\usepackage{mathtools,amsmath}
\usepackage{amssymb,amsthm} % assumes amsmath package installed
\newtheorem{theorem}{Theorem}
\newtheorem{corollary}{Corollary}
\usepackage{graphicx}
\usepackage{tikz}
\usetikzlibrary{arrows}
\usepackage{verbatim}
\usepackage{float}
\usepackage{tikz}
\usetikzlibrary{automata,positioning}
\usepackage{pgfplots}
\usetikzlibrary{shapes,arrows}
\usetikzlibrary{arrows,calc,positioning}
\tikzset{
	block/.style = {draw, rectangle,
		minimum height=1cm,
		minimum width=1.5cm},
	input/.style = {coordinate,node distance=1cm},
	output/.style = {coordinate,node distance=4cm},
	arrow/.style={draw, -latex,node distance=2cm},
	pinstyle/.style = {pin edge={latex-, black,node distance=2cm}},
	sum/.style = {draw, circle, node distance=1cm},
}
\usepackage{xcolor}
\usepackage{mdframed}
\usepackage[shortlabels]{enumitem}
\usepackage{indentfirst}
\usepackage{hyperref}
\usepackage{wrapfig}
\usepackage{tcolorbox}

%% Работа с языком
\usepackage[utf8]{inputenc}
\usepackage[english]{babel}
\usepackage{booktabs}
\usepackage{pgfplots}
\pgfplotsset{width=9cm, height=5cm, compat=1.9}

\renewcommand{\thesubsection}{\thesection.\alph{subsection}}

\newtheoremstyle{problemstyle}  % <name>
{3pt}                                               % <space above>
{3pt}                                               % <space below>
{\normalfont}                               % <body font>
{}                                                  % <indent amount}
{\bfseries\itshape}                 % <theorem head font>
{\normalfont\bfseries:}         % <punctuation after theorem head>
{.5em}                                          % <space after theorem head>
{}                                                  % <theorem head spec (can be left empty, meaning `normal')>
\theoremstyle{problemstyle}

\newtheorem{problem}{Problem}[section]

% Define solution environment
\newenvironment{solution}
{\textit{\textbf{Solution:}}}
{}
\renewcommand{\qed}{\quad\qedsymbol}


\newcommand{\R}{\mathbb{R}}
\newcommand{\E}{\mathbb{E}}
\renewcommand{\P}{\mathbb{P}}
\DeclareMathOperator{\Var}{\mathbb{V}ar}
\DeclareMathOperator{\AsVar}{As.\mathbb{V}ar}
\DeclareMathOperator{\Cov}{\mathbb{C}ov}
\DeclareMathOperator{\Corr}{\mathbb{C}orr}
\DeclareMathOperator{\sign}{sign}
\DeclareMathOperator{\tr}{tr}
\DeclareMathOperator{\rank}{rank}
\DeclareMathOperator{\argmax}{argmax}
\DeclareMathOperator{\MSE}{MSE}





% магия для автоматических гиперссылок задача-решение
\newlist{myenum}{enumerate}{3}
% \newcounter{problem}[chapter] % нумерация задач внутри глав


% \setcounter{tocdepth}{1} % в оглавление оставляем уровень 1




\author{Nick Averyanov, Sasha Plakhin}
\title{Financial math problems solutions}

\begin{document}

\maketitle

\newpage
\tableofcontents{}

\newpage
\section{Binomial model}

\section{Itô's lemma}

Let $dX_t = \mu_t dt + \sigma_t dW_t$ and $f(t, x) \in C^{1,2}$. Itô's lemma:
\begin{equation}
df(t, X_t) = \bigg(\cfrac{\partial f}{\partial t} + \mu_t \cfrac{\partial f}{\partial x}  + \cfrac{1}{2}\sigma^2_t \cfrac{\partial f^2}{\partial x^2} \bigg) dt + \sigma_t \cfrac{\partial f}{\partial x} dW_t 
\end{equation}
Remember that this is just a short notation for the integral form. 

\begin{problem}
$h(\cdot)$ -- is a harmonic function if:
$$
\sum_{i=1}^n \cfrac{\partial^2 h}{\partial x_i^2} = 0.
$$
$h(\cdot) $ -- is a subharmonic function if:
$$
\sum_{i=1}^n \cfrac{\partial^2 h}{\partial x_i^2} \geq 0.
$$
Prove that for independent  Wiener processes $W_1, \ldots, W_n$ and a processes $X$ is defined by the formula: $X(t) = h(W_1(t), \ldots, W_n(t))$. Show that if $h$ is harmonic (subharmonic) $\Rightarrow$  $X$ is a martingale (submartingale).
\end{problem}

\begin{solution}
	Applying multidimensional version of Ito's lemma to the function $h$ we can obtain:
	$$
	dX = \sum_{i=1}^n \cfrac{\partial h}{\partial x_i} dW_i + \cfrac{1}{2} \sum_{i=1}^n \cfrac{\partial^2 h}{\partial x_i^2} dt
	$$
	Equivalently, we can rewrite the equation as:
	$$
	h(W_1(t), \ldots, W_n(t)) ) = h(\boldsymbol{0}) + \int_0^t  \sum_{i=1}^n \cfrac{\partial h}{\partial x_i} dW_i + \int_0^t\cfrac{1}{2} \sum_{i=1}^n \cfrac{\partial^2 h}{\partial x_i^2} dt
	$$
	\begin{multline*}
		\E[X(t)| \mathcal{F}_s] = h(\boldsymbol{0})  + \E\bigg[ \int_0^s  \sum_{i=1}^n \cfrac{\partial h}{\partial x_i} dW_i + \int_0^s\cfrac{1}{2} \sum_{i=1}^n \cfrac{\partial^2 h}{\partial x_i^2} dt + \int_s^t  \sum_{i=1}^n \cfrac{\partial h}{\partial x_i} dW_i + \int_s^t\cfrac{1}{2} \sum_{i=1}^n \cfrac{\partial^2 h}{\partial x_i^2} dt | \mathcal{F}_s \bigg ] =  \\
		X(s) + \E \bigg[ \int_s^t  \sum_{i=1}^n \cfrac{\partial h}{\partial x_i} dW_i + \int_s^t\cfrac{1}{2} \sum_{i=1}^n \cfrac{\partial^2 h}{\partial x_i^2} dt | \mathcal{F}_s \bigg ] = X(s) + \E \bigg[ \int_s^t\cfrac{1}{2} \sum_{i=1}^n \cfrac{\partial^2 h}{\partial x_i^2} dt | \mathcal{F}_s \bigg ]
	\end{multline*}
	Looking at the last term we see that there is a sum of second order partial derivatives. If $h$ is harmonic we have $\E[X(t)|\mathcal{F}_s] = X(s)$ a.s. If  $h$ is subharmonic we have $\E[X(t)|\mathcal{F}_s] \geq X(s)$ a.s. And thus we have proven the statement.

\end{solution}

\begin{problem}
	Show that $dW_1 dW_2 = 0 $ for two independent Brownian motions.
\end{problem}
Let $\Delta W_i(t_k)  = W_i(t_k) - W_i(t_{k-1})$.
Define $Q_n$:
$$
Q_n = \sum_{k=1}^n  \Delta W_1(t_k) \Delta W_2(t_k),
$$
where $0 = t_0 < t_1 < \ldots <t_n = t$. Now we just have to show that $Q_n$ converges to $0$ in $L^2$ as the norm of the partiotion goes to $0$. We have:
$$
\E[Q_n] = \E \sum_{k=1}^n  \Delta W_1(t_k) \Delta W_2(t_k)  =  \sum_{k=1}^n  \E[ \Delta W_1(t_k) \Delta W_2(t_k)] = 0
$$
\begin{multline*}
\E[Q_n^2] = \Var[Q_n]  = \Var\bigg[ \sum_{k=1}^n  \Delta W_1(t_k) \Delta W_2(t_k)  \bigg] = 
 \sum_{k=1}^n \Var[\Delta W_1(t_k) \Delta W_2(t_k)] = \\
 \sum_{k=1}^n  \E[\Delta W_1(t_k)^2 \Delta W_2(t_k)^2] = \sum_{k=1}^n  \E[\Delta W_1(t_k)^2] \E[\Delta W_2(t_k)^2]  = \\ \sum_{k=1}^n  (\Delta t_k)^2 \leq 
 \max_k \{\Delta t_k \} \sum_{k=1}^n \Delta t_k =  \max_{k \in \{, \ldots n\}} \{\Delta t_k \} t \to 0, n \to \infty
\end{multline*}

\section{Martingales}

\section{Partial differential equations}
\subsection{Probabilistic representation of the Cauchy problem solutions}

\noindent In this subsection, we show some simple examples on the topic of probabilistic representation of the Cauchy problem solutions.

Some theory. Given the conditions:
$$
\frac{\partial F}{\partial t}(t, x)  + \mu(t, x) \frac{\partial F}{\partial x}(t, x)  + \frac{1}{2} \sigma^2(t, x) \frac{\partial^2 F}{\partial x^2 }(t, x)
-rF(t, x)  = 0,
\;\;\;\; F(T, x) = \Phi(x)
$$
$$
dX_s = \mu(s, X_s) ds + \sigma(s, X_s)dW_s, \;\;\;  X_t = x
$$
We can write down the Feynman-Kac formula for the solution:
$$
F(t, x) = e^{-r(T-t)}\E_{t, x}[\Phi(X_T)]
$$
\begin{problem}
Use a stochastic representation result in order to solve the following boundary value problem in the domain  $[0, T] \times \R$:
$$
\frac{\partial F}{\partial t} + \mu x \frac{\partial F}{\partial x} + \frac{1}{2} \sigma^2 x^2 \frac{\partial^2 F}{\partial x^2 } = 0
$$    
$$
F(T,x) = \ln(x^2) 
$$

Here $\mu$ and $\sigma$ are assumed to be known constants.

\end{problem}

\begin{solution}
    Applying Feynman–Kac formula we have
$$
F(t,x) = \E[\ln X_T^2| X_t = x]
$$

where

$$
dX_s = \mu X_s ds + \sigma X_s dW_s
$$
$$
X_t = x
$$

Applying Ito formula, we can investigate the process $Z_s = \ln X_s^2$
$$
d Z_s = \frac{2}{X_s}dX_s - \sigma^2 ds = \frac{2\mu X_s ds + 2 \sigma X_s dW_s}{X_s} - \sigma^2 ds = (2\mu - \sigma^2)ds + 2\sigma dW_s\\
$$

Thus we have the equation

$$
d \ln X^2_s = (2\mu - \sigma^2)ds + 2\sigma dW_s
$$

We can rewrite it in the following form

$$
\ln X_T^2= \ln X_t^2 + \int\limits_t^T (2\mu - \sigma^2) ds + \int\limits_{t}^{T}2\sigma dW_s = \ln X_t^2 + (2\mu - \sigma^2) (T-t) + 2\sigma (W_T - W_t)
$$

Thus we have the solution:

$$
F(t,x) = \E[\ln X_T^2| X_t = x] = \ln(x^2) + (2\mu - \sigma^2) (T-t)
$$
\end{solution}

\begin{problem}
Consider the following boundary value problem in the domain
$[0,T] \times R$.

$$ 
\frac{\partial F}{\partial t} + \mu(t,x) \frac{\partial F}{\partial x} + \frac{1}{2}\sigma^2(t,x) \frac{\partial^2 F}{\partial x^2} + k(t,x) = 0
$$
$$
F(T,x) = \Phi(x)
$$

Here $\mu, \sigma, k$ and $\Phi$ are assumed to be known functions.

Prove that this problem has the stochastic representation formula.
$$
F(t, x) = \E[\Phi(X_T)| X_t = x] + \int\limits_t^T E[k(s, X_s)|X_t = x]ds
$$

where 
$$
dX_s = \mu(s, X_s) ds + \sigma(s, X_s) dW_s
$$
$$
X_t = x
$$
\end{problem}

\begin{solution}
Considering the process $Z_s = F(s, X_s)$ and applying Ito formula, we have

\begin{multline*}
dZ_s = \frac{\partial F }{\partial s} (s, X_s) ds + \frac{\partial F }{\partial x} (s, X_s) dX_s + \frac{1}{2} \sigma^2(s, X_s) \frac{\partial^2 F}{\partial x^2} (s, X_s)ds = \\ =\underbrace{\Big( \frac{\partial F }{\partial s} (s, X_s) + \mu(s, X_s) \frac{\partial F }{\partial x} (s, X_s)  + \frac{1}{2} \sigma^2(s, X_s) \frac{\partial^2 F}{\partial x^2} (s, X_s)\Big)}_{= -k(s, X_s), \text{ assuming that $F$ actually solves PDE} }ds + \sigma(s, X_s)\frac{\partial F }{\partial x} (s, X_s) dW_S \\
\end{multline*}

We can rewrite this equation in the following form

$$
F(t, X_t) = F(T, X_T) + \int\limits_{t}^{T}k(s,X_s)ds - \int\limits_{t}^{T}\sigma(s, X_s)\frac{\partial F }{\partial x} (s, X_s) dW_S
$$

If the process $\sigma(s, X_s)\frac{\partial F }{\partial x} (s, X_s)$ is in $\mathcal{L}^2 $, taking expected values we have the proof(remember that intitial value $X_t = x$, $F(T,x) = \Phi(x)$ and expected value of Wiener integral is equal to zero)

$$
F(t,x) = \E[\Phi(X_T)|X_t = x] + \int\limits_{t}^{T} \E[k(s,X_s)|X_t = x]ds
$$
\end{solution}




\section{Stochastic differential equations}






\section{Black-Scholes model}
Suppose that we have an underlying asset with price $S_t$. Its dynamics follows geometric Brownian motion:
$$
dS_t = \mu S_tdt + \sigma S_t d\bar{W}_t
$$
We also have a riskless asset (bond) with the following dynamics ($r$ is the interest rate):
$$
dB_t = rB_tdt
$$


 Some assumptions about the market are also necessary (inifinite liquidity etc). 
Suppose we now want to price some derivative with payment obligation $\mathcal{X} = \Phi(S_T)$ at time moment T. Then under risk-neutral measure $\mathbb{Q}$ we have no-arbitrage price:
$$
\Pi_t = F(t, S_t) = e^{-r(T-t)}\E^{\mathbb{Q}}_{t, s} [\Phi(S_T)]
$$

For a price of a  european call option the formula is:
\begin{eqnarray}
F(t, s) &=& s \cdot N[d_1(t, s)] - e^{-r(T-t)}K \cdot N[d_2(t, s)] \\
d_1(t, s) &=&  \cfrac{1}{\sigma\sqrt{T-t} }
	\bigg( 
	\log\bigg(\cfrac{s}{K}\bigg) + \bigg( r + \cfrac{1}{2} \sigma^2 \bigg)(T-t) 
	\bigg) \\
d_2(t, s) &=& d_1(t, s) - \sigma \sqrt{T-t} \\
N(x) &=& \int_{-\infty}^{x} \cfrac{1}{\sqrt{2\pi}} e^{-t^2/2}dt
\end{eqnarray}

\begin{problem}
	Consider basic Black-Scholes model with $T$-payment obligation $\mathcal{X} = \Phi(S_T)$. Let $\Pi_t$ be a no-arbitrage price.
	\begin{enumerate}
		\item[a)] Show that under martingale measure $\mathbb{Q}$:
		$$
		d\Pi_t = r \Pi_t dt + g(t) dW_t
		$$ 
		
		\item[b)]
		Show that $\cfrac{\Pi_t}{B_t}$ is a martingale.
	\end{enumerate}
	\begin{solution}
		\item[a)] Under measure $\mathbb{Q}$ we know that $S_t$ is a geometric Brownian motion with drift $r$. We also know that $F$ is an actual solution for the equation:
		$$
		\frac{\partial F}{\partial t}  +  r \frac{\partial F}{\partial s} + \frac{1}{2} \sigma^2 \frac{\partial^2 F}{\partial s^2 }
		-r F = 0
		\;\;\;\; F(T, s) = \Phi(s)
		$$  
		Using these facts and having $g(t) \equiv \sigma \cfrac{\partial F}{\partial s}$:
		\begin{multline}
		d\Pi_t = dF = \bigg(\frac{\partial F}{\partial t}  +  r \frac{\partial F}{\partial s} + \frac{1}{2} \sigma^2 \frac{\partial^2 F}{\partial s^2 }\bigg)dt + \sigma \frac{\partial F}{\partial s} dW_t = rF dt+ \sigma \frac{\partial F}{\partial s} dW_t  = r\Pi_t dt + g(t) dW_t
		\end{multline}
		
		\item[b)] We are going to apply Itô's lemma to the function $e^{-rt}B_0^{-1}\Pi_t$:
		\begin{multline}
		d(e^{-rt}B_0^{-1}\Pi_t) = -re^{-rt}B_0^{-1}\Pi_t dt +e^{-rt}B_0^{-1}d\Pi_t = \\
		 -re^{-rt}B_0^{-1}\Pi_t dt + e^{-rt}B_0^{-1}(r\Pi_t dt + g(t) dW_t) = 
		 e^{-rt}B_0^{-1} \sigma \frac{\partial F}{\partial s} dW_t
		\end{multline}
		This stochastic differntial contains zero drift component $\Rightarrow$ $\cfrac{\Pi_t}{B_t}$ is a martingale.
	\end{solution}
\end{problem}

\begin{problem}
	Consider a derivative with payoff $\Phi(S_T) = \log(S_T)$. Find its no-arbitragee price.
	
	\begin{solution}
		For simplicity let us assume $S_0 \equiv 1$ (it does not affect the answer). In fact we only have to compute conditional expectation with respect to a risk-neutral measure $\mathbb{Q}$
		\begin{multline}
		\E^{\mathbb{Q}}_{t, s}[\log S_T] = 
		\E^{\mathbb{Q}}\bigg[ 
		\bigg(r  - \cfrac{\sigma^2}{2}\bigg)T + \sigma W_T \bigg| 
		\bigg(r  - \cfrac{\sigma^2}{2}\bigg)t + \sigma W_t = \log(s)
		\bigg] = \\
		\bigg(r  - \cfrac{\sigma^2}{2}\bigg)T  +
		\E^{\mathbb{Q}}\bigg[ \sigma W_T \bigg| 
		\sigma W_t = \log(s) - \bigg(r  - \cfrac{\sigma^2}{2}\bigg)t 
		\bigg] = 
			\bigg(r  - \cfrac{\sigma^2}{2}\bigg)(T-t) +\log(s) 
		\end{multline}
		Here we used the fact that $\sigma W_t$ is a martingale. Then we can write down the price:
		\begin{equation}
		\Pi_t = e^{-r(T-t)}\E^{\mathbb{Q}}_{t, s}[\log S_T]  = e^{-r(T-t)}\bigg(\bigg(r  - \cfrac{\sigma^2}{2}\bigg)(T-t) +\log(s)\bigg) 
		\end{equation}
	\end{solution}
\end{problem}

\begin{problem}
		Price a european put-option (knowing the Black-Scholes formula for the call).
	
	\begin{solution}
		We know that if we buy call (denote its price by $C$) and sell a put (denote its price by $P$) at time $T$, we get $\max\{S-K, 0\}$ for call and $-\max\{K-S, 0\}$  for put.  Formally:
		\begin{equation}
		C - P = \max\{S-K, 0\} - \max\{K-S, 0\} = S - K% \Rightarrow  C - P - S = -K
		\end{equation}
		For a more time moment $t < T$ we should also discount the strike price. Now let's rearrange the terms and get the formula:
		\begin{multline}
		P(t, s)  = C (t, s)- s+e^{-r(T-t)}K = 
		 s \cdot N[d_1(t, s)] - e^{-r(T-t)}K \cdot N[d_2(t, s)]  - s + e^{-r(T-t)}K = \\
		 = e^{-r(T-t)}K \cdot N[-d_2(t, s)]  - s \cdot N[-d_1(t, s)]
		\end{multline}
	\end{solution}
\end{problem}

\begin{problem}
	Dynamics of the stock price in dollars follows geometric Brownian motion:
	\begin{equation}
	dS_t = \mu S_t dt + \sigma S_t d\bar{W}^1_t.
	\end{equation}
	While the euro-dollar exchange rate also follows geometric Brownian motion:
		\begin{equation}
	dY_t = \beta Y_t dt + \delta Y_t d\bar{W}^2_t.
	\end{equation}
	Both Brownian motions are independent of each other. There is also a derivative which pays $\ln(Z_T^2)$ at time $T$, where $Z_t$ is a price of a stock in euro. A risk-free rate in euros is equal to $r$.  Find the arbitrage free price for the derivative. 
	
	\begin{solution}
		In fact we can rewrite $Z_t$ as $Y_t \cdot S_t$. Le'ts now find $dZ_t$ using Ito's product rule:
		\begin{multline}
		dZ_t = d(Y_t \cdot S_t) = Y_t dS_t + S_t dY_t + dY_t dS_t =
		 Y_t dS_t + S_t dY_t  = \\
		 (\mu + \beta) Z_t dt + \sigma Z_t d\bar{W}^1_t + \delta Z_t d\bar{W}^2_t = (\mu + \beta) Z_t dt + \sqrt{\sigma^2 + \delta^2} Z_t d \tilde{W}_t
		\end{multline}
				Here $\tilde{W}_t$ is also a Brownian motion. Now we can use a result from one of the previous problems and get:
		\begin{equation}
		\Pi_t = e^{-r(T-t)}\E^{\mathbb{Q}}[\log(Z_t^2)] =
			e^{-r(T-t)}((2r  - (\sigma^2 + \delta^2))(T-t) + 2\log(z)) 
		\end{equation}
	\end{solution}
	
\end{problem}



\end{document}

